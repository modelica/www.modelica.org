%sample file for Modelica 2005 Conference paper

\documentclass[11pt,a4paper,twocolumn]{article}
\usepackage{graphicx}
% do not change this lines
\usepackage[latin1]{inputenc}    %% european ���� characters can be used
\usepackage{times, mathptm}      %% recommended for readable pdf
\pagestyle{empty}                %% no page numbers!
\usepackage{geometry}            %% please don't change geometry settings!
\geometry{left=20mm, right=20mm, top=25mm, bottom=25mm, noheadfoot}
\parindent0pt

% begin the document
\begin{document}
\thispagestyle{empty}

\title{\textbf{Implementation of a Modelica Library\\
  for Simulation of Refrigeration Systems}}
\author{Torge Pfafferott \quad Gerhard Schmitz\\
Technical University Hamburg-Harburg, Department of Technical Thermodynamics\\
Denickestr. 17, 21075 Hamburg}
\date{} % <--- leave date empty
\maketitle\thispagestyle{empty} %% <-- you need this for the first page

\abstract{
The physical modelling and transient simulation of
refrigeration systems can be useful within the specification,
development, integration and optimisation.
Therefore, a model library for vapour compression cycles has been implemented.
The library is based on the free Modelica library ThermoFluid and contains basic correlations for
heat and mass transfer and pressure drop, partial components for control volumes and
flow resistances and advanced ready-to-use models for all relevant
components of refrigeration systems like pipes, heat exchangers,
compressor, expansion devices and accumulator.
}

\emph{Keywords: refrigeration; compression cycle; simulation; thermofluid; CO2; R134a}

\section{Introduction}

The modeling and simulation of refrigeration systems is of interest 
for several problems:

\section{Library for refrigeration systems}

The aim of the modelling is to implement a library with physical based
models of components of refrigeration systems. At the moment the
library enables investigations with two refrigerants (CO$_2$, R134a). But
the realised structure allows the extension of the library by other
refrigerants.

\subsection{ThermoFluid library}

The implemented refrigeration library is based on the free Modelica library ThermoFluid
\cite{eborn}, \cite{tum}, \cite{thermofluid}. The
ThermoFluid library, especially its base classes and partial
components, offers a good base for the modelling of refrigeration systems with
respect to the implementation of the three balance equations and the
method of discretisation. 

\section{Transient simulation of a CO$_2$-system}

In the following, results of the transient simulation of the above mentioned CO$_2$-system are presented.
The results are compared with data of a start up of the
system and following step changes in compressor speed as shown in
Figure \ref{fig5}.

\begin{figure}[h]
%uncomment next line to include a graphic file
%\centerline{\includegraphics[width=6cm, angle=-90]{fig5.eps}}
%and comment out next line
\centerline{\framebox[6cm]{\rule{0cm}{3.5cm} figure example}}
\caption{Step changes in compressor speed and run of air inlet
temperature at the evaporator in the experiment; set as boundary
condition of simulation run}
\label{fig5}
\end{figure}

\begin{thebibliography}{00}
\addcontentsline{toc}{chapter}{References}

\bibitem{eborn} Eborn J. On Model Libraries for Thermo-hydraulic
Applications. Lund, Sweden: PhD thesis, Department of Automatic
control, Lund Institute of Technology, 2001.

\bibitem{tum}Tummescheit H. Design and Implementation of Object-Oriented Model Libraries
using Modelica. Lund, Sweden: PhD thesis, Department of Automatic
control, Lund Institute of Technology, 2002.

\bibitem{thermofluid} Tummescheit H, Eborn J. Chemical Reaction
Modeling with ThermoFluid/MF and MultiFlash. In: Proceedings of the 2th
Modelica Conference 2002, Oberpfaffenhofen, Germany, Modelica
Association, 18-19 March 2002.
\end{thebibliography}


\end{document}
